% Created 2018-05-16 Wed 10:42
% Intended LaTeX compiler: pdflatex
\documentclass[final,12pt,a4paper]{article}

\usepackage{graphicx}
\usepackage{amssymb}
\usepackage[margin=0.6in]{geometry}
\usepackage{booktabs}
\usepackage{xcolor}
\usepackage{sourcecodepro}
\usepackage{url}
\usepackage{listings}
\usepackage[utf8]{inputenc}
\usepackage[english]{babel}
\usepackage{multirow}
\usepackage{textcomp}
\usepackage{caption}
\usepackage{hyperref}
\usepackage{sourcecodepro}
\usepackage{booktabs}
\usepackage{array}
\usepackage{listings}
\usepackage{graphicx}
\usepackage[english]{babel}
\usepackage[scale=2]{ccicons}
\usepackage{url}
\usepackage{relsize}
\usepackage{amsmath}
\usepackage{bm}
\usepackage{wasysym}
\usepackage{ragged2e}
\usepackage{textcomp}
\usepackage{pgfplots}
\usepgfplotslibrary{dateplot}
\setsansfont[BoldFont={Source Sans Pro Semibold},Numbers={OldStyle}]{Source Sans Pro}
\lstdefinelanguage{Julia}%
{morekeywords={abstract,struct,break,case,catch,const,continue,do,else,elseif,%
end,export,false,for,function,immutable,mutable,using,import,importall,if,in,%
macro,module,quote,return,switch,true,try,catch,type,typealias,%
while,<:,+,-,::,/},%
sensitive=true,%
alsoother={$},%
morecomment=[l]\#,%
morecomment=[n]{\#=}{=\#},%
morestring=[s]{"}{"},%
morestring=[m]{'}{'},%
}[keywords,comments,strings]%
\lstset{ %
backgroundcolor={},
basicstyle=\ttfamily\scriptsize,
breakatwhitespace=true,
breaklines=true,
captionpos=n,
commentstyle=\color{black},
extendedchars=true,
frame=n,
keywordstyle=\color{black},
language=R,
rulecolor=\color{black},
showspaces=false,
showstringspaces=false,
showtabs=false,
stepnumber=2,
stringstyle=\color{gray},
tabsize=2,
}
\renewcommand*{\UrlFont}{\ttfamily\smaller\relax}
\author{Pedro Bruel}
\date{\today}
\title{Autotuning: D-Optimal Designs}
\hypersetup{
 pdfauthor={Pedro Bruel},
 pdftitle={Autotuning: D-Optimal Designs},
 pdfkeywords={},
 pdfsubject={},
 pdfcreator={Emacs 25.3.1 (Org mode 9.1.13)}, 
 pdflang={English}}
\begin{document}

\maketitle
\tableofcontents


\section{Autotuning with D-Optimal Designs and Analysis of Variance}
\label{sec:org9539d2d}
\begin{enumerate}
\item Use \texttt{optFederov} to find 24 experiments for the full model:
\begin{align*}
    Y = & \; y\_component\_number + 1 / y\_component\_number + \\
        & \; vector\_length + lws\_y + 1 / lws\_y + \\
        & \; load\_overlap + temporary\_size + \\
        & \; elements\_number + 1 / elements\_number + \\
        & \; threads\_number + 1 / threads\_number
\end{align*}
\item Use \texttt{aov} to fit the full model, spending the 24 evaluations:
\begin{align*}
      time\_per\_pixel = & \; y\_component\_number + 1 / y\_component\_number + \\
                        & \; vector\_length + lws\_y + 1 / lws\_y + \\
                        & \; load\_overlap + temporary\_size + \\
                        & \; elements\_number + 1 / elements\_number + \\
                        & \; threads\_number + 1 / threads\_number
\end{align*}
\item Identify the most significant factors from the ANOVA summary. In this
case, they are \(vector\_length\) and \(lws\_y\).
\item Use the fitted model to predict the best \(time\_per\_pixel\) value in the
entire dataset
\item Prune the dataset using the predicted best values for \(vector\_length\) and \(lws\_y\)
\item Use \texttt{optFederov} to find 24 experiments for the pruned model. If there are less
than or exactly 24 candidates, use the full candidate set.
\begin{align*}
    Y = & \; y\_component\_number + 1 / y\_component\_number + \\
        & \; load\_overlap + temporary\_size + \\
        & \; elements\_number + 1 / elements\_number + \\
        & \; threads\_number + 1 / threads\_number
\end{align*}
\item Use \texttt{aov} to fit the pruned model, spending the 24 evaluations:
\begin{align*}
      time\_per\_pixel = & \; y\_component\_number + 1 / y\_component\_number + \\
                        & \; load\_overlap + temporary\_size + \\
                        & \; elements\_number + 1 / elements\_number + \\
                        & \; threads\_number + 1 / threads\_number
\end{align*}
\item Identify the most significant factors from the ANOVA summary. In this
case, they are \(y\_component\_number\) and \(threads\_number\).
\item Use the fitted model to predict the best \(time\_per\_pixel\) value in the
entire dataset
\item Prune the dataset using the predicted best values for \(y\_component\_number\) and
\(threads\_number\)
\item Use \texttt{optFederov} to find 24 experiments for the pruned model. If there are less
than or exactly 24 candidates, use the full candidate set.
\begin{align*}
    Y = & \; load\_overlap + temporary\_size + \\
        & \; elements\_number + 1 / elements\_number
\end{align*}
\item Use \texttt{aov} to fit the pruned model, spending the 24 evaluations:
\begin{align*}
      time\_per\_pixel = & \; load\_overlap + temporary\_size + \\
                        & \; elements\_number + 1 / elements\_number
\end{align*}
\item Identify the most significant factors from the ANOVA summary. In this
case, it is \(elements\_number\)
\item Use the fitted model to predict the best \(time\_per\_pixel\) value in the
entire dataset
\item Prune the dataset using the predicted best values for \(elements\_number\)
\item Use \texttt{optFederov} to find 24 experiments for the pruned model. If there are less
than or exactly 24 candidates, use the full candidate set.
\begin{align*}
    Y = load\_overlap + temporary\_size
\end{align*}
\item Use \texttt{aov} to fit the pruned model, spending the 24 evaluations:
\begin{align*}
      time\_per\_pixel = load\_overlap + temporary\_size
\end{align*}
\item Use the fitted model to predict the best \(time\_per\_pixel\) value in the
entire dataset
\item Compare the predicted \(time\_per\_pixel\) with the global optimum
\end{enumerate}
\section{Results}
\label{sec:org6bebcd2}
\subsection{Comparing Strategies}
\label{sec:orgc639a14}
\begin{center}
\includegraphics[width=.9\linewidth]{../img/comparison_histogram.pdf}
\end{center}

% latex table generated in R 3.4.4 by xtable 1.8-2 package
% Wed May 16 10:42:15 2018
\begin{table}[ht]
\centering
\begingroup\small
\begin{tabular}{lrrrrrrrrrrr}
  \hline
 & RS & LHS & GS & GSR & GA & LM & LMB & RQ & DOPT & DLM & DLMT \\ 
  \hline
Min. & 1.00 & 1.00 & 1.00 & 1.00 & 1.00 & 1.01 & 1.01 & 1.01 & 1.38 & 1.01 & 1.01 \\ 
  1st Qu. & 1.03 & 1.09 & 1.35 & 1.07 & 1.02 & 1.01 & 1.01 & 1.01 & 1.64 & 1.01 & 1.01 \\ 
  Median & 1.08 & 1.19 & 1.80 & 1.19 & 1.09 & 1.01 & 1.03 & 1.01 & 1.64 & 1.01 & 1.01 \\ 
  Mean & 1.10 & 1.17 & 6.46 & 1.23 & 1.12 & 1.02 & 1.03 & 1.02 & 1.68 & 1.01 & 1.01 \\ 
  3rd Qu. & 1.18 & 1.24 & 6.31 & 1.33 & 1.19 & 1.01 & 1.03 & 1.01 & 1.64 & 1.01 & 1.01 \\ 
  Max. & 1.39 & 1.52 & 124.76 & 3.16 & 1.65 & 3.77 & 3.80 & 2.06 & 2.91 & 1.08 & 1.01 \\ 
  Mean Pt. & 120.00 & 98.92 & 22.17 & 120.00 & 120.00 & 119.00 & 104.81 & 119.00 & 120.00 & 54.85 & 54.84 \\ 
  Max Pt. & 125.00 & 125.00 & 106.00 & 120.00 & 120.00 & 119.00 & 106.00 & 119.00 & 120.00 & 56.00 & 56.00 \\ 
   \hline
\end{tabular}
\endgroup
\caption{Summary statistics} 
\end{table}
\subsection{Comparing Models}
\label{sec:org23bad93}
\begin{center}
\includegraphics[width=.9\linewidth]{../img/model_comparison.pdf}
\end{center}
\subsection{Checking Accuracy}
\label{sec:org1ca0c82}
To verify the ``accuracy'' of the selected metrics, I adapted the experiment
scripts to check for each removed model variable in the actual \texttt{aov} summary.
Those initial choices seem to match in most cases with the variables identified
as most relevant by the \texttt{aov} summary, as shown below.

As described previously, at each step a group of variables is removed from the
model based on their "score", that is, the "Pr(>F)" value in the \texttt{aov} summary.
I selected at most two variables at each of the three steps, based on preliminary
visual analysis of the \texttt{aov} summaries.

To measure how accurate those initial selections were I checked at each step if
the \(n\) selected variables were in the \(n\) most relevant variables in that
step's \texttt{aov} summary. If that was the case I incremented a step-specific
counter. The counters were updated for 1000 iterations and then divided by 1000.
This value represents the accuracy of the static selection in comparison with
the values that would be selected if each individual \texttt{aov} summary was analysed.

\begin{center}
\includegraphics[width=0.45\textwidth]{../img/doptaov_accuracy.pdf}
\includegraphics[width=0.45\textwidth]{../img/lmbm_accuracy.pdf}
\end{center}

\begin{center}
\includegraphics[width=0.45\textwidth]{../img/dlmt_accuracy.pdf}
\end{center}
\end{document}
